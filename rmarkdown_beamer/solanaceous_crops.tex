% Options for packages loaded elsewhere
\PassOptionsToPackage{unicode}{hyperref}
\PassOptionsToPackage{hyphens}{url}
%
\documentclass[
  ignorenonframetext,
  aspectratio=169]{beamer}
\usepackage{pgfpages}
\setbeamertemplate{caption}[numbered]
\setbeamertemplate{caption label separator}{: }
\setbeamercolor{caption name}{fg=normal text.fg}
\beamertemplatenavigationsymbolsempty
% Prevent slide breaks in the middle of a paragraph
\widowpenalties 1 10000
\raggedbottom
\setbeamertemplate{part page}{
  \centering
  \begin{beamercolorbox}[sep=16pt,center]{part title}
    \usebeamerfont{part title}\insertpart\par
  \end{beamercolorbox}
}
\setbeamertemplate{section page}{
  \centering
  \begin{beamercolorbox}[sep=12pt,center]{part title}
    \usebeamerfont{section title}\insertsection\par
  \end{beamercolorbox}
}
\setbeamertemplate{subsection page}{
  \centering
  \begin{beamercolorbox}[sep=8pt,center]{part title}
    \usebeamerfont{subsection title}\insertsubsection\par
  \end{beamercolorbox}
}
\AtBeginPart{
  \frame{\partpage}
}
\AtBeginSection{
  \ifbibliography
  \else
    \frame{\sectionpage}
  \fi
}
\AtBeginSubsection{
  \frame{\subsectionpage}
}
\usepackage{lmodern}
\usepackage{amssymb,amsmath}
\usepackage{ifxetex,ifluatex}
\ifnum 0\ifxetex 1\fi\ifluatex 1\fi=0 % if pdftex
  \usepackage[T1]{fontenc}
  \usepackage[utf8]{inputenc}
  \usepackage{textcomp} % provide euro and other symbols
\else % if luatex or xetex
  \usepackage{unicode-math}
  \defaultfontfeatures{Scale=MatchLowercase}
  \defaultfontfeatures[\rmfamily]{Ligatures=TeX,Scale=1}
\fi
\usetheme[]{Frankfurt}
\usecolortheme{beaver}
% Use upquote if available, for straight quotes in verbatim environments
\IfFileExists{upquote.sty}{\usepackage{upquote}}{}
\IfFileExists{microtype.sty}{% use microtype if available
  \usepackage[]{microtype}
  \UseMicrotypeSet[protrusion]{basicmath} % disable protrusion for tt fonts
}{}
\makeatletter
\@ifundefined{KOMAClassName}{% if non-KOMA class
  \IfFileExists{parskip.sty}{%
    \usepackage{parskip}
  }{% else
    \setlength{\parindent}{0pt}
    \setlength{\parskip}{6pt plus 2pt minus 1pt}}
}{% if KOMA class
  \KOMAoptions{parskip=half}}
\makeatother
\usepackage{xcolor}
\IfFileExists{xurl.sty}{\usepackage{xurl}}{} % add URL line breaks if available
\IfFileExists{bookmark.sty}{\usepackage{bookmark}}{\usepackage{hyperref}}
\hypersetup{
  pdftitle={Solanaceous crops},
  pdfauthor={Deependra Dhakal},
  hidelinks,
  pdfcreator={LaTeX via pandoc}}
\urlstyle{same} % disable monospaced font for URLs
\newif\ifbibliography
\setlength{\emergencystretch}{3em} % prevent overfull lines
\providecommand{\tightlist}{%
  \setlength{\itemsep}{0pt}\setlength{\parskip}{0pt}}
\setcounter{secnumdepth}{-\maxdimen} % remove section numbering
\usepackage{booktabs}
\usepackage{longtable}
\usepackage{array}
\usepackage{multirow}
\usepackage{wrapfig}
\usepackage{float}
\usepackage{colortbl}
\usepackage{pdflscape}
\usepackage{tabu}
\usepackage{threeparttable}
\usepackage{threeparttablex}
\usepackage[normalem]{ulem}
\usepackage{makecell}
\usepackage{xcolor}
\usepackage{tikz} % required for image opacity change
\usepackage[absolute,overlay]{textpos} % for text formatting

% this font option is amenable for beamer
\setbeamerfont{caption}{size=\tiny}

% redefine section page format
\AtBeginSection{
   \frame{\sectionpage}
}

\makeatletter
\setbeamertemplate{section page}
{
  \begingroup
    \centering
%    {\usebeamerfont{section name}\usebeamercolor[fg]{section name}\sectionname~\insertsectionnumber}
    \vskip1em\par
    \begin{beamercolorbox}[sep=12pt,center,colsep=-4bp,rounded=true,shadow=\beamer@themerounded@shadow]{section title}
      \usebeamerfont{section title}\insertsection\par
    \end{beamercolorbox}
  \endgroup
}
\makeatother

\title{Solanaceous crops}
\author{Deependra Dhakal}
\date{2019}

\begin{document}
\frame{\titlepage}

\begin{frame}[allowframebreaks]
  \tableofcontents[hideallsubsections]
\end{frame}
\hypertarget{tomato-lycopersicon-esculentum}{%
\section{\texorpdfstring{Tomato ( \emph{Lycopersicon
esculentum})}{Tomato ( Lycopersicon esculentum)}}\label{tomato-lycopersicon-esculentum}}

\begin{frame}{Background}
\protect\hypertarget{background}{}
\begin{itemize}
\tightlist
\item
  Poor man's orange
\item
  Immediate ancestor: \emph{L. esculentum} Var. \emph{cerasiforme}
\item
  Varieties can broadly be classified into:

  \begin{itemize}
  \tightlist
  \item
    Determinate: Terminal portion of stem ending in flowering shoot;
    Each node produces flower
  \item
    Indeterminate: Terminal portion of stem is growing and growth is
    unchecked. Flowers are produced every 3rd internode.
  \end{itemize}
\item
  Tomato is winter season crop for terai region and summer crop for mid
  hills
\item
  Color pigments for tomato:

  \begin{itemize}
  \tightlist
  \item
    Red: Lycopene
  \item
    Yellow: Carotenoid
  \item
    Prolycopene: Tangerine
  \end{itemize}
\end{itemize}
\end{frame}

\begin{frame}{}
\protect\hypertarget{section}{}
\begin{itemize}
\tightlist
\item
  Tomato receives attention from the National Seed Vision with regard as
  crop that have importance for import substitution (Srijana Hybrid and
  Lapsigede) and export promotion.
\item
  Asia-Pacific Seed Association (APSA) has initiated harmonization
  process of seed regulations at the regional level in Asia. APSA has
  selected 10 crops, which include maize, rice, sunflower, cabbage,
  cauliflower, cucumber, eggplant, hot pepper, tomato and watermelon for
  harmonization. Although, Nepal has not fully participated in the
  harmonization process yet.
\end{itemize}
\end{frame}

\begin{frame}{Climate}
\protect\hypertarget{climate}{}
\begin{itemize}
\tightlist
\item
  Optimum temperature: \(20^\circ C\) for night and \(30^\circ C\) for
  day
\item
  Red and yellow pigments develop at \(10-25^\circ C\)
\item
  It has been reported that lycopene formation is favoured at
  temperatures ranging from \(16-21^\circ C\) and unfavoured at
  temperature values above \(30^\circ C\)
\item
  Lycopene is destroyed above \(40^\circ C\).
\end{itemize}
\end{frame}

\begin{frame}{Soil}
\protect\hypertarget{soil}{}
\begin{itemize}
\tightlist
\item
  pH of 6.7 is ideal.
\item
  Soil of Sarlahi district is very well suited because of loamy texture
  with high organic matter content.
\end{itemize}
\end{frame}

\begin{frame}{Cultivation}
\protect\hypertarget{cultivation}{}
\begin{itemize}
\tightlist
\item
  Seedling is raised in beds of 2 x 1m. A single bed suffices for a
  ropani of planting.
\item
  Spacing:

  \begin{itemize}
  \tightlist
  \item
    Pusa ruby: 60 x 60 sqcm
  \item
    Pusa early dwarf: 60 x 60 sqcm
  \item
    Roma: 60 x 45 sqcm
  \item
    Monprecos: 75 x 60 sqcm
  \end{itemize}
\end{itemize}
\end{frame}

\begin{frame}{Manuring and fertilization}
\protect\hypertarget{manuring-and-fertilization}{}
\begin{itemize}
\tightlist
\item
  1 mt FYM and 5:3:4 kg NPK/ropani
\end{itemize}
\end{frame}

\begin{frame}{Varieties}
\protect\hypertarget{varieties}{}
\footnotesize

\begin{itemize}
\tightlist
\item
  A detailed listing of tomato varieties can be assessed at
  Wikipedia\footnote<.->{\url{https://en.wikipedia.org/wiki/List_of_tomato_cultivars}}
\item
  National Seed Vision, 2013-25\footnote<.->{\url{http://extwprlegs1.fao.org/docs/pdf/nep147056.pdf}}
  enlists NCL, Lapsigede, Monprecos, and Pusarubi varieties for variety
  maintenance and seed production pocket/zone specification. Midhills,
  inner terai and terai regions are mentioned.
\item
  Hybrid variety of tomato (Srijana) was registered in 2010.
\item
  Pusa Ruby

  \begin{itemize}
  \tightlist
  \item
    Indeterminate type (IARI)
  \item
    Early maturing (60 DAT)
  \item
    Released: 2046 BS
  \end{itemize}
\item
  Pusa Early Dwarf

  \begin{itemize}
  \tightlist
  \item
    Determinate
  \item
    Early maturing
  \item
    Terai and mid-hills
  \end{itemize}
\item
  Coimbatore-1

  \begin{itemize}
  \tightlist
  \item
    Tamil Nadu Agricultural University
  \item
    Determinate type
  \end{itemize}
\end{itemize}
\end{frame}

\begin{frame}{}
\protect\hypertarget{section-1}{}
\begin{itemize}
\tightlist
\item
  Mangala

  \begin{itemize}
  \tightlist
  \item
    F1 hybrid determinate type
  \item
    For hills
  \end{itemize}
\item
  Best of all

  \begin{itemize}
  \tightlist
  \item
    Indeterminate
  \item
    IARI
  \end{itemize}
\item
  Roma

  \begin{itemize}
  \tightlist
  \item
    IARI, determinate type
  \item
    Early (Midhills and terai)
  \item
    Suitable for processing
  \item
    Released: 2051
  \end{itemize}
\item
  Monprecos

  \begin{itemize}
  \tightlist
  \item
    Indeterminate type, OP
  \item
    Medium maturing
  \item
    Tolerant to late blight disease
  \item
    Mid and high hills
  \item
    Released: 2051
  \end{itemize}
\end{itemize}
\end{frame}

\begin{frame}{}
\protect\hypertarget{section-2}{}
\begin{itemize}
\tightlist
\item
  NCL-1 (CL-1131)

  \begin{itemize}
  \tightlist
  \item
    OP
  \item
    Head and wilt resistant
  \item
    Released: 2051 BS
  \item
    Indeterminate type
  \end{itemize}
\item
  Srijana

  \begin{itemize}
  \tightlist
  \item
    National hybrid
  \item
    Maturity: 70-80 DAT
  \item
    Bacterial wilt resistant
  \item
    Released: 2066 BS
  \end{itemize}
\item
  Swaraksha

  \begin{itemize}
  \tightlist
  \item
    Hybrid
  \item
    Registered: 2066 BS
  \end{itemize}
\end{itemize}
\end{frame}

\begin{frame}{}
\protect\hypertarget{section-3}{}
\begin{itemize}
\tightlist
\item
  NS-2535

  \begin{itemize}
  \tightlist
  \item
    Hybrid
  \item
    Released: 2066 BS
  \item
    Production: 140-150 mt/ha
  \end{itemize}
\item
  Dalila

  \begin{itemize}
  \tightlist
  \item
    Hybrid
  \item
    Terai, mid hills and high hills
  \item
    Released: 2067 BS
  \end{itemize}
\item
  Others: Haryana Selection 102 (OP), Ceres, NS-53, Ahmita, Nova, Xiw,
  Sens, NS-515, VLHH3, Spectra, Gaurav-555, Marina, Astra-717, Savera,
  Eureka, NS-719
\item
  Varieties registered in 2067 BS: Madhuri, Makis, Jamuna, Opel
\item
  OPVs: Roma (Determinate), Pusa Ruby (Indeterminate), Monprecos
  (Indeterminate), NCL-1
\end{itemize}
\end{frame}

\begin{frame}{Time of planting}
\protect\hypertarget{time-of-planting}{}
\begin{itemize}
\tightlist
\item
  High-hills:

  \begin{itemize}
  \tightlist
  \item
    Chaitra-Baisakh (Sowing)
  \item
    Baisakh-Jestha (Transplanting)
  \end{itemize}
\item
  Mid-hills:

  \begin{itemize}
  \tightlist
  \item
    Falgun-Chaitra (Sowing)
  \item
    Chaitra-Baisakh (Transplanting)
  \end{itemize}
\item
  Lower-hills:

  \begin{itemize}
  \tightlist
  \item
    Magh-Falgun (Sowing)
  \item
    Falgun-Chaitra (Transplanting)
  \end{itemize}
\item
  Terai:

  \begin{itemize}
  \tightlist
  \item
    Ashwin-Kartik (Sowing)
  \item
    Kartik-Mangsir (Transplanting)
  \end{itemize}
\end{itemize}
\end{frame}

\begin{frame}{}
\protect\hypertarget{section-4}{}
\begin{itemize}
\tightlist
\item
  Indeterminate types are also called staking type
\item
  Yield almost double than unstaked
\item
  Fruits are attractive, uniform and high quality
\item
  Insect-pest and diease also minimized
\item
  Stakign utmost necessary when growing in rainy season
\end{itemize}
\end{frame}

\begin{frame}{}
\protect\hypertarget{section-5}{}
Yield: 1000-1500 kg
\end{frame}

\begin{frame}{Seed production}
\protect\hypertarget{seed-production}{}
\begin{itemize}
\tightlist
\item
  Extraction has 3 methods:

  \begin{enumerate}
  [a.]
  \tightlist
  \item
    Fermentation method
  \item
    Alkali treatment
  \end{enumerate}

  \begin{itemize}
  \tightlist
  \item
    Slime mass is scooped out and treated with 300 gm of washing soda in
    4 ltr of boiling water in equal volume.
  \end{itemize}

  \begin{enumerate}
  [a.]
  \setcounter{enumi}{2}
  \tightlist
  \item
    Acid treatment
  \end{enumerate}

  \begin{itemize}
  \tightlist
  \item
    Slime mass + Acid HCL (0.1 N) @ 75 ml/12 kg of fruit material
  \end{itemize}
\item
  Seed yield: 100-120 kg/ha
\end{itemize}
\end{frame}

\begin{frame}{Disease}
\protect\hypertarget{disease}{}
\footnotesize

\begin{itemize}
\tightlist
\item
  Damping off

  \begin{itemize}
  \tightlist
  \item
    Pythium, Phytophthora, Sclerotinia (Active in low temperatures)
  \item
    Rhizoctonia (Severe in high temperatures)
  \end{itemize}
\item
  Control

  \begin{itemize}
  \tightlist
  \item
    Sparse planting, light irrigation, decomposed manure
  \item
    Burning of 6-12 inches of farm trash on seed bed
  \item
    Drenching of nursery bed with Dithane M-45 @ 3 gm/ltr of water.
  \item
    Seed treatment with Agrosan GN, Captan or Thiram @ 2.5 gm/kg of
    seed.
  \end{itemize}
\item
  Fusarium wilt

  \begin{itemize}
  \tightlist
  \item
    \emph{Fusarium oxysporum} f.sp. \emph{lycopersici}
  \item
    Symptom: Upward and inward wilting of leaves
  \item
    Advanced stage: Browning of vascular system seen in cross-section of
    lower stem
  \end{itemize}
\item
  Control

  \begin{itemize}
  \tightlist
  \item
    Seed treatment with Bavistin 2.5 gm/kg of seed
  \item
    Crop rotation for long period
  \item
    Trash burning
  \end{itemize}
\item
  Bacterial diseases

  \begin{itemize}
  \tightlist
  \item
    Bacterial wilt ( \emph{Pseudomonas solanacearum})
  \item
    Symptom: Wilting, stunting and yellowing of foliage and collapse of
    plants
  \end{itemize}
\end{itemize}
\end{frame}

\hypertarget{brinjal-aubergine-eggplant}{%
\section{Brinjal (AUbergine,
Eggplant)}\label{brinjal-aubergine-eggplant}}

\begin{frame}{}
\protect\hypertarget{section-6}{}
\begin{itemize}
\tightlist
\item
  Self pollinated
\item
  Winter season crop for terai and summer and rainy season crop for
  hills
\item
  Can't produce fruit below temperature of \(17^\circ C\) and above
  \(35^\circ C\)
\item
  Silt loam and clay loam are the most preferred soil types
\end{itemize}
\end{frame}

\begin{frame}{Varieties}
\protect\hypertarget{varieties-1}{}
\begin{itemize}
\tightlist
\item
  Nurki

  \begin{itemize}
  \tightlist
  \item
    OPV (60 - 65 DAT)
  \item
    Terai and mid-hills
  \item
    Registered: 2051 BS
  \item
    Bears fruits in cluster
  \end{itemize}
\item
  Arka Keshav

  \begin{itemize}
  \tightlist
  \item
    OPV (70-75 DAT)
  \item
    Registered: 2066
  \item
    Terai and mid-hills
  \end{itemize}
\end{itemize}
\end{frame}

\begin{frame}{}
\protect\hypertarget{section-7}{}
\begin{itemize}
\tightlist
\item
  Early maturing: Pusa purple long (PPL), Pusa kranti
\item
  Mid season: Pusa cluster, Nurki
\item
  Late variety: Sarlahi green
\end{itemize}
\end{frame}

\begin{frame}{}
\protect\hypertarget{section-8}{}
\begin{itemize}
\tightlist
\item
  Seed rate: 25-30 gm/ropani
\item
  Spacing

  \begin{itemize}
  \tightlist
  \item
    Tall varieties (Sarlahi Green, Nurki, Pusa Kranti): 75 x 45 sqcm
  \item
    Medium and short varieties (PPL, Pusa cluster): 60 x 45 sqcm
  \end{itemize}
\end{itemize}
\end{frame}

\begin{frame}{}
\protect\hypertarget{section-9}{}
\begin{itemize}
\tightlist
\item
  Yield: 1500-2000 kg/ropani
\end{itemize}
\end{frame}

\begin{frame}{}
\protect\hypertarget{section-10}{}
\begin{itemize}
\tightlist
\item
  Average seed yield: 100-120 kg/ha
\end{itemize}
\end{frame}

\begin{frame}{}
\protect\hypertarget{section-11}{}
\begin{itemize}
\tightlist
\item
  Pusa purple long is a photoinsensitive type
\item
  Nurki (Hard skinned)
\end{itemize}
\end{frame}

\hypertarget{hot-pepper-and-sweet-pepper}{%
\section{Hot pepper and sweet
pepper}\label{hot-pepper-and-sweet-pepper}}

\begin{frame}{}
\protect\hypertarget{section-12}{}
\begin{itemize}
\tightlist
\item
  \emph{Capsicum frutescens} and \emph{Capsicum annum}
\item
  Plant annual and fruits borne singly, \emph{Capsicum annum}
\item
  Plant perennial and fruits borne in goups, \emph{C. frutescens}
\end{itemize}
\end{frame}

\begin{frame}{}
\protect\hypertarget{section-13}{}
\begin{itemize}
\tightlist
\item
  Seed rate: 3 kg/ha
\end{itemize}
\end{frame}

\begin{frame}{Cultivars}
\protect\hypertarget{cultivars}{}
\begin{enumerate}
\tightlist
\item
  Sweet pepper
\end{enumerate}

\begin{itemize}
\tightlist
\item
  Californe (California wonder)

  \begin{itemize}
  \tightlist
  \item
    OPV (80-90 DAT)
  \item
    16-20 mt/ha yield
  \item
    Suitable for terai, mid-hills and high-hills
  \item
    Registered: 2051 BS
  \end{itemize}
\item
  Sagar

  \begin{itemize}
  \tightlist
  \item
    OPV (65-75 DAT)
  \item
    76.8 mt/ha
  \end{itemize}
\end{itemize}
\end{frame}

\hypertarget{bibliography}{%
\section{Bibliography}\label{bibliography}}

\begin{frame}{References}
\protect\hypertarget{references}{}
\end{frame}

\end{document}
