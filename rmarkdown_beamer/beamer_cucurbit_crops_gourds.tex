% Options for packages loaded elsewhere
\PassOptionsToPackage{unicode}{hyperref}
\PassOptionsToPackage{hyphens}{url}
%
\documentclass[
  ignorenonframetext,
  aspectratio=169]{beamer}
\usepackage{pgfpages}
\setbeamertemplate{caption}[numbered]
\setbeamertemplate{caption label separator}{: }
\setbeamercolor{caption name}{fg=normal text.fg}
\beamertemplatenavigationsymbolsempty
% Prevent slide breaks in the middle of a paragraph
\widowpenalties 1 10000
\raggedbottom
\setbeamertemplate{part page}{
  \centering
  \begin{beamercolorbox}[sep=16pt,center]{part title}
    \usebeamerfont{part title}\insertpart\par
  \end{beamercolorbox}
}
\setbeamertemplate{section page}{
  \centering
  \begin{beamercolorbox}[sep=12pt,center]{part title}
    \usebeamerfont{section title}\insertsection\par
  \end{beamercolorbox}
}
\setbeamertemplate{subsection page}{
  \centering
  \begin{beamercolorbox}[sep=8pt,center]{part title}
    \usebeamerfont{subsection title}\insertsubsection\par
  \end{beamercolorbox}
}
\AtBeginPart{
  \frame{\partpage}
}
\AtBeginSection{
  \ifbibliography
  \else
    \frame{\sectionpage}
  \fi
}
\AtBeginSubsection{
  \frame{\subsectionpage}
}
\usepackage{lmodern}
\usepackage{amssymb,amsmath}
\usepackage{ifxetex,ifluatex}
\ifnum 0\ifxetex 1\fi\ifluatex 1\fi=0 % if pdftex
  \usepackage[T1]{fontenc}
  \usepackage[utf8]{inputenc}
  \usepackage{textcomp} % provide euro and other symbols
\else % if luatex or xetex
  \usepackage{unicode-math}
  \defaultfontfeatures{Scale=MatchLowercase}
  \defaultfontfeatures[\rmfamily]{Ligatures=TeX,Scale=1}
\fi
\usetheme[]{Frankfurt}
\usecolortheme{beaver}
% Use upquote if available, for straight quotes in verbatim environments
\IfFileExists{upquote.sty}{\usepackage{upquote}}{}
\IfFileExists{microtype.sty}{% use microtype if available
  \usepackage[]{microtype}
  \UseMicrotypeSet[protrusion]{basicmath} % disable protrusion for tt fonts
}{}
\makeatletter
\@ifundefined{KOMAClassName}{% if non-KOMA class
  \IfFileExists{parskip.sty}{%
    \usepackage{parskip}
  }{% else
    \setlength{\parindent}{0pt}
    \setlength{\parskip}{6pt plus 2pt minus 1pt}}
}{% if KOMA class
  \KOMAoptions{parskip=half}}
\makeatother
\usepackage{xcolor}
\IfFileExists{xurl.sty}{\usepackage{xurl}}{} % add URL line breaks if available
\IfFileExists{bookmark.sty}{\usepackage{bookmark}}{\usepackage{hyperref}}
\hypersetup{
  pdftitle={Cucurbit crops: gourds},
  pdfauthor={Deependra Dhakal},
  hidelinks,
  pdfcreator={LaTeX via pandoc}}
\urlstyle{same} % disable monospaced font for URLs
\newif\ifbibliography
\setlength{\emergencystretch}{3em} % prevent overfull lines
\providecommand{\tightlist}{%
  \setlength{\itemsep}{0pt}\setlength{\parskip}{0pt}}
\setcounter{secnumdepth}{-\maxdimen} % remove section numbering
\usepackage{booktabs}
\usepackage{longtable}
\usepackage{array}
\usepackage{multirow}
\usepackage{wrapfig}
\usepackage{float}
\usepackage{colortbl}
\usepackage{pdflscape}
\usepackage{tabu}
\usepackage{threeparttable}
\usepackage{threeparttablex}
\usepackage[normalem]{ulem}
\usepackage{makecell}
\usepackage{xcolor}
\usepackage{tikz} % required for image opacity change
\usepackage[absolute,overlay]{textpos} % for text formatting

% this font option is amenable for beamer
\setbeamerfont{caption}{size=\tiny}

% % redefine section page format
% \defbeamertemplate{section page}{mine}[1][]{%
%   \begin{centering}
%     {\usebeamerfont{section name}\usebeamercolor[fg]{section name}#1}
%     \vskip1em\par
%     \begin{beamercolorbox}[sep=12pt,center]{part title}
%       \usebeamerfont{section title}\insertsection\par
%     \end{beamercolorbox}
%   \end{centering}
% }

\title{Cucurbit crops: gourds}
\author{Deependra Dhakal}
\date{2019}

\begin{document}
\frame{\titlepage}

\begin{frame}[allowframebreaks]
  \tableofcontents[hideallsubsections]
\end{frame}
\hypertarget{introduction}{%
\section{Introduction}\label{introduction}}

\begin{frame}{Background}
\protect\hypertarget{background}{}
\begin{itemize}
\tightlist
\item
  Consist several genera indigenous to Africa, India, Asia, and the
  Americas.
\item
  Generally eaten when immature, some edible as they ripe.
\item
  Cylindrical, discoid, or bottle-shaped.
\item
  Seeds found in the central cavity.
\item
  Mostly grown in Asia and parts of Africa.
\end{itemize}
\end{frame}

\begin{frame}{}
\protect\hypertarget{section}{}
\begin{table}

\caption{\label{tab:gourds-intro}Nomenclature of gourds}
\centering
\begin{tabular}[t]{ll}
\toprule
Crop & Scientific Name\\
\midrule
Bottle gourd & \textit{Lagenaria siceraria} (Mol.)\\
Bitter gourd & \textit{Momordica charantia} (L.)\\
Pointed gourd & \textit{Trichosanthes dioica} (Roxb.)\\
Snake gourd & \textit{Trichosanthes cucumerina}\\
Sponge gourd & \textit{Luffa cylindrica}\\
\addlinespace
 & \textit{Luffa aegyptiaca} (Mill)\\
Ash gourd & \textit{Benincasa hispida} (Thunb.)\\
Chayote & \textit{Sechium edule} (Swartz)\\
\bottomrule
\end{tabular}
\end{table}
\end{frame}

\begin{frame}{Uses}
\protect\hypertarget{uses}{}
\begin{itemize}
\tightlist
\item
  Sponge gourd upon maturing is fibrous and can be used as rough sponge
  or scrub pad.
\item
  Airtight container (Calabash gourd)
\item
  Can be stored after slice drying for later use as vegetable.
\item
  Some are even eaten raw.
\end{itemize}
\end{frame}

\hypertarget{bottle-gourd}{%
\section{Bottle gourd}\label{bottle-gourd}}

\begin{frame}{Introduction}
\protect\hypertarget{introduction-1}{}
\begin{itemize}
\tightlist
\item
  Common in Nepal and India, but grown in Ethiopia, Africa, Central
  America and warmer regions.
\item
  Fruits' shape resemble to that of a bottle.
\item
  Mostly yellowish-green or cream colored, relatively soft in texture,
  with white pulp and large white seeds.
\item
  Used for making sweets and pickles.
\item
  Beneficial for jaundice patients, in those with digestive problems,
  cough, nightblindness, etc.
\item
  Monoecious annual vine with large oxalate oval leaves and branched
  tendrils spreading or climbing 3-15m.
\item
  Foliage pubescent and emits musky odor when bruised.
\item
  Flowers large white, borne on slender peduncles, open in the evening
  and may remain open until the middle of the following day.
\item
  Fruits vary in shape and sizes(10-90cm in length)
\end{itemize}
\end{frame}

\begin{frame}{Cultivation: Climate}
\protect\hypertarget{cultivation-climate}{}
\begin{itemize}
\tightlist
\item
  Tropical crop(hot and humid climate)
\item
  Optimum temperature: \(24-27^\circ C\)
\item
  Highly sensitive to photoperiod
\item
  Short day and humid climate promote femaleness
\end{itemize}
\end{frame}

\begin{frame}{Cultivation: Soil}
\protect\hypertarget{cultivation-soil}{}
\begin{itemize}
\tightlist
\item
  Best on sandy loam with high OM content
\item
  Well drained
\item
  pH: 6-7
\item
  River bed farming is also successful
\end{itemize}
\end{frame}

\begin{frame}{Cultivation: Propagation}
\protect\hypertarget{cultivation-propagation}{}
\begin{itemize}
\tightlist
\item
  Seeds(450-500/100g)
\item
  Seed retains viability for long periods(upto 34 years)
\end{itemize}
\end{frame}

\begin{frame}{Cultivation: Cultural Practices}
\protect\hypertarget{cultivation-cultural-practices}{}
\begin{itemize}
\tightlist
\item
  Shallow rooted crop
\item
  Fine bed preparation
\item
  Direct sowing on raised beds, in furrows, or trenches or pits
\item
  Two seeds per hill on either sides of raised beds or furrows
\item
  Spacing: 2.5*2m or 2-3*1.5m
\item
  Pit: 90*90*60cm deep, filled with FYM and topsoil
\item
  3-4 seeds per pit
\item
  45 t/ha FYM
\item
  40-60:40-60:60-80 kg/ha NPK
\item
  Nitrogen in split doses(Basal and at vining stage)
\item
  Pruning to exert favorable effects on fruit yield
\end{itemize}
\end{frame}

\begin{frame}{Cultivation: Interculture}
\protect\hypertarget{cultivation-interculture}{}
\begin{itemize}
\tightlist
\item
  Shallow interculture operations needed
\item
  Hand weeding(2-3 times)
\item
  Pre-emergence herbicide: linuron @ 0.5kg/ha or alachlor 2.5 kg/ha
\end{itemize}
\end{frame}

\begin{frame}{Cultivation: Irrigation}
\protect\hypertarget{cultivation-irrigation}{}
\begin{itemize}
\tightlist
\item
  Immediately after seed sowing to promote germination
\item
  High humidity promotes prolific bearing
\item
  Irrigation every 3-4 days in warm weather
\end{itemize}
\end{frame}

\begin{frame}{Cultivation: Maturity and Harvesting}
\protect\hypertarget{cultivation-maturity-and-harvesting}{}
\begin{itemize}
\tightlist
\item
  Fruit setting increased by maleic hydrazide spray(400 ppm), boron(3
  ppm) and calcium (20 ppm)
\item
  Ready for harvest approx. 60 DAS
\item
  12-15 days after fruit set to reach marketable size
\item
  Picking every 4-5 days
\end{itemize}
\end{frame}

\hypertarget{bitter-gourd}{%
\section{Bitter gourd}\label{bitter-gourd}}

\begin{frame}{Cultivation: Climate and Soil}
\protect\hypertarget{cultivation-climate-and-soil}{}
\begin{itemize}
\tightlist
\item
  Grown in tropical or subtropical climates, but warm and hot climate
  best for growth
\item
  Most resistant to low temperatures than other cucurbits
\item
  Optimum temp: \(24^\circ C-27^\circ C\)
\item
  Short day promotes female flower production
\item
  Grow on sandy to loamy soils
\item
  Optimum pH: 6-6.7
\end{itemize}
\end{frame}

\begin{frame}{Cultivation: Propagation}
\protect\hypertarget{cultivation-propagation-1}{}
\begin{itemize}
\tightlist
\item
  Seed propagated commercially
\item
  Have hard seed coat, impermeable to water
\item
  Heavy seeds, and retain viability for upto 45 years
\item
  Cultural Practices
\item
  Seeds sown in raised beds or pits
\item
  Raised beds: 1.2-1.5m wide beds, 60 cm wide furrows between beds
\item
  Spacing: 1.5-2m between rows, 60-120cm between plants
\end{itemize}
\end{frame}

\begin{frame}{}
\protect\hypertarget{section-1}{}
\begin{itemize}
\tightlist
\item
  Pit planting: 60*60*45 cm deep, plants spaced at 1.8-2.4cm apart with
  two seeds per hill
\item
  Seed rate: 4.5-6kg/ha
\item
  Training or staking required, mostly when grown as rainy season crop
\item
  Bowers erected 6 ft. from ground level using wooden poles. Strong
  poles 10 ft. in length and 4 cm in diameter fixed 45 cm deep in soil,
  5 m apart. All poles connected by wire. 16-gauge wires stretched over
  the 10-gauge wires 60 cm apart, crosswise.
\item
  Kniffens may also be prepared by wooden poles and wire
\end{itemize}
\end{frame}

\begin{frame}{Cultivation: Fertilization and manuring}
\protect\hypertarget{cultivation-fertilization-and-manuring}{}
\begin{itemize}
\tightlist
\item
  60-80 kg N, 70-90 kg P, 60-70 kg K per hectare
\item
  Nitrogen best applied in split doses(latter half @initial fruit-set
  stage)
\end{itemize}
\end{frame}

\begin{frame}{Cultivation: Irrigation}
\protect\hypertarget{cultivation-irrigation-1}{}
\begin{itemize}
\tightlist
\item
  First, Immediately after sowing
\item
  At regular intervals of 5-7 days
\end{itemize}
\end{frame}

\begin{frame}{Cultivation: Weed management}
\protect\hypertarget{cultivation-weed-management}{}
\begin{itemize}
\tightlist
\item
  Interculture to ensure proper spread of crop
\item
  Weedicides: alachlor and butachlor @2.5 kg ai/ha
\end{itemize}
\end{frame}

\begin{frame}{Cultivation: Maturity and harvest}
\protect\hypertarget{cultivation-maturity-and-harvest}{}
\begin{itemize}
\tightlist
\item
  Crop takes about 55-110 days from seed sowing to first harves
\item
  Fruits picked while still tender
\item
  Picking done every 2-3 days
\item
  Yellowing of fruit indicated ripening, which is unfit for consumption
\item
  Fruits storable for 35 days under cool and shady conditions
\end{itemize}
\end{frame}

\hypertarget{pointed-gourds}{%
\section{Pointed gourds}\label{pointed-gourds}}

\begin{frame}{Cultivation: Propagation}
\protect\hypertarget{cultivation-propagation-2}{}
\begin{itemize}
\tightlist
\item
  Crop is dioecious
\item
  Vegetatively propagated with vine cuttings and root suckers
\item
  Seed propagation has poor germination
\item
  50\% plants may be non-fruiting(male) types
\item
  10-12\% male plants is adequate
\item
  Generally, early maturing varieties with fewer nodes bear pistillate
  flowers
\item
  Cuttings transplanted in August in upland and November in riverbeds
\item
  Spacing: 2 * 2 m
\item
  Root cuttings transpalnted in second half of October
\item
  Vine cuttings of length 1-1.5m cut and folded in a figure of ``8'' or
  ring, and planted in pits with mixture of FYM and soil and buried 10
  cm deep.
\end{itemize}
\end{frame}

\hypertarget{snake-gourds}{%
\section{Snake gourds}\label{snake-gourds}}

\hypertarget{sponge-gourds}{%
\section{Sponge gourds}\label{sponge-gourds}}

\hypertarget{ash-gourds}{%
\section{Ash gourds}\label{ash-gourds}}

\hypertarget{cultivation-outline}{%
\section{Cultivation outline}\label{cultivation-outline}}

\begin{frame}{Cultivation outline}
\begin{table}

\caption{\label{tab:cultivation-outline}Cultivation outline of gourds}
\centering
\fontsize{6}{8}\selectfont
\begin{tabular}[t]{>{\raggedright\arraybackslash}p{5em}>{\raggedright\arraybackslash}p{6em}>{\raggedright\arraybackslash}p{8em}>{\raggedright\arraybackslash}p{8em}>{\raggedright\arraybackslash}p{8em}>{\raggedright\arraybackslash}p{5em}>{\raggedright\arraybackslash}p{5em}>{\raggedright\arraybackslash}p{5em}>{\raggedright\arraybackslash}p{5em}}
\toprule
Crop & Varieties & Sowing time(High hills) & Sowing time(Mid hills) & Sowing time(Terai hills) & FYM & NPK/ropani & Spacing & Seed required/ropani\\
\midrule
\rowcolor{gray!6}  Bottle gourd & N.S-421 &  & Falgun-Ashar & Poush-Jestha & 1500 & 2:2:1 & 200 x 200 & 50-100 g\\
 & Summer prolific long & Baisakh-Jestha & Falgun-Chaitra & Magh-Jestha &  &  & 200 x 200 & 50-100 g\\
\rowcolor{gray!6}  Sponge gourd & Kantipure, Jyapu & Baisakh-Jestha & Falgun-Jestha & Magh-Jestha & 500 & 2:1:1 & 300 x 300 & 50-100 g\\
 & Pusa Chillo & Baisakh-Jestha & Falgun-Jestha & Magh-Jestha &  &  &  & \\
\rowcolor{gray!6}  Bitter gourd & Green & Baisakh-Jestha & Falgun-Chaitra & Magh-Jestha & 1500 & 10:6:3 & 150 x 100 & 100 g\\
\addlinespace
 & Coimbatore long & Baisakh-Jestha & Falgun-Chaitra & Magh-Jestha &  &  & 150 x 100 & 100 g\\
\rowcolor{gray!6}   & Creeper & Baisakh-Jestha & Falgun-Jestha & Poush-Jestha &  &  & 150 x 100 & 100 g\\
\bottomrule
\end{tabular}
\end{table}
\end{frame}

\hypertarget{bibliography}{%
\section{Bibliography}\label{bibliography}}

\begin{frame}{References}
\protect\hypertarget{references}{}
\end{frame}

\end{document}
